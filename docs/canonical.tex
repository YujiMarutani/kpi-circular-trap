\documentclass[11pt]{article}

% ============================================================
% KPI Circular Trap and Spiral Escape via Decentralized Learning Pipelines
% v1.0 -- Canonical Release
%
% This document represents a growth-oriented institutional artifact.
% It is intentionally incomplete and expected to be negated,
% restructured, or superseded by future versions.
%
% The DOI associated with this work denotes a continuously verified
% process rather than a finalized conclusion.
% ============================================================

\usepackage{amsmath, amssymb, amsthm}
\usepackage{hyperref}
\usepackage{geometry}
\usepackage{graphicx}
\usepackage{enumitem}
\usepackage{authblk}

\geometry{margin=1in}

\title{
KPI Circular Trap and Spiral Escape via \\
Decentralized Learning Pipelines \\
{\large v1.0 -- Canonical Release}
}

\author{
Yuji Marutani \\
Independent Researcher \\
ORCID: 0009-0000-9692-9399
}

\date{February 2026}

\begin{document}

\maketitle

\begin{abstract}
Advanced economies face persistent productivity stagnation despite increasingly sophisticated performance indicators.
This paper defines the root cause of this stagnation as the \emph{KPI Circular Trap}: a self-referential loop in which low-resolution consensus metrics suppress high-resolution individual sovereignty.

We formalize individual sovereignty using the $\chi$-model and propose a decentralized learning pipeline connecting open educational resources (OCW), AI-assisted sensemaking (NotebookLM), and decentralized validation mechanisms (DAO).
This pipeline enables individuals to bypass institutional filtering and reconstruct knowledge at higher semantic resolution.

The central contribution of this work is not only the presentation of a theoretical framework, but the implementation of a verification apparatus, \emph{CIFF} (Continuous Integration for Fundamental Freedom), which evaluates whether knowledge artifacts are generated under human sovereignty rather than autonomous AI production.
CIFF operationalizes this distinction through asymmetry-based verification, examining semantic deviation, responsibility density in code, and iterative self-negation across versions.

Furthermore, this paper adopts a \emph{growth-oriented publication model} in which the DOI represents a continuously verified institutional artifact rather than a finalized conclusion.
Version~1.0 constitutes the canonical baseline of this system, explicitly positioning the work as a dynamic protocol aiming to satisfy the condition $d\chi/dt \geq 0$ over time.

This work should therefore be read not as a completed answer, but as the initial deployment of a human-centric verification system for knowledge production in the age of AI.

\end{abstract}
\noindent
\textbf{Canonical DOI (v1.0):} \url{https://doi.org/10.5281/zenodo.18470162}

\newpage
\section{Introduction: The KPI Circular Trap}

Productivity metrics such as standardized test scores, academic rankings, and performance indicators are designed to improve efficiency and accountability.
However, when widely adopted, these metrics tend to converge toward simplified consensus representations, reducing semantic resolution at the individual level.

We refer to this phenomenon as the \emph{KPI Circular Trap}: a closed loop in which individuals are trained, evaluated, and optimized according to low-resolution indicators that ultimately suppress non-conforming but potentially high-impact cognitive trajectories.

\section{The $\chi$-Model of Sovereignty}

We define individual sovereignty not as autonomy from systems, but as the signal-to-noise ratio (SNR) of meaningful action under increasing consensus bandwidth.
Formally, we introduce the following model:

\begin{equation}
\chi = \frac{\mathrm{SNR}_{\text{Sovereign}}}{\mathrm{BW}_{\text{Consensus}} \cdot \log (N+1)}
\end{equation}

The objective of the proposed system is not to maximize $\chi$ at a single point in time, but to ensure the non-negative growth condition:
\begin{equation}
\frac{d\chi}{dt} \geq 0
\end{equation}

\section{CIFF: Continuous Integration for Fundamental Freedom}

CIFF is introduced in this version as a conceptual verification apparatus.
Its role is to distinguish human-sovereign knowledge production from autonomous AI generation through asymmetry-based analysis.
Implementation details and formal verification modules will be expanded in v1.1 and later versions.

\section{Growth-Oriented Publication Model}

This work adopts a growth-oriented publication model in which each version represents a verifiable state in a continuous process of theoretical and institutional evolution.
Sections left minimal or abstract in v1.0 are intentionally preserved as future sites of negation, refinement, and structural reconfiguration.


\end{document}



